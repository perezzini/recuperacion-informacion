Resulta cada vez más frecuente la creación de disciplinas y ramificaciones de las ciencias que vienen para resolver problemáticas puntuales. Recuperación de información puede ser un buen ejemplo de ello y como muchos otros campos se enfocan en atacar el problema generando modelos basados en conocimientos previos para llegar a una posible solución. Actualmente podemos ver como la interrelación con otras áreas de estudio e investigación como puede ser la minería de datos o la inteligencia artificial que aportan soluciones integrales para mejorar resultados anteriores y que aportan valor a nuevos desafíos que van surgiendo en el campo. \\
Tomando como caso de estudio a los motores de búsquedas, vemos que estos sistemas deben lidiar con problemas adicionales que un simple sistema de recuperación con una base de información acotada y con datos homogéneos. Dar respuesta a búsquedas sobre colecciones de objetos con datos estructurados, semi-estructurados y no estructurados que presentan un dinamismo de cambio sin precedentes. El hecho que se pueda recuperar datos con diversos formatos y en diferentes idiomas genera mayores desafíos para todo el proceso de indexación y recuperación.  \\
Revisamos el proceso de analizar y procesar la gigantesca cantidad de datos en la web mediante programas denominados crawlers que recorren la red a través de los hiperenlaces presentes en los sitios web. \\
Luego un algoritmo que parece haber encontrado una buena solución al problema de generar un orden de relevancia sobre los resultados de búsquedas. \\
Por último, un acercamiento en el estado del arte con los sistemas recomendadores, los  cuales aportan una mejor experiencia para el usuario satisfaciendo necesidades de información de forma implícita y aprovechando de la relación entre los usuarios para entregar una respuesta con mayor precisión. Otros trabajos existentes que aprovechan esas relaciones entre los usuarios del sistema para obtener un beneficio mutuo es el desarrollo de sistemas de recuperación colaborativos de información que vienen a facilitar el proceso de búsqueda y recuperación cuando se vuelve una tarea ardua ya sea porque hay mucha información sobre el tema o se necesita recuperar información de varios temas que tiene cierta relación con un trabajo dado. Este último enfoque no fue tratado en el presente trabajo pero consideramos importante mencionarlo porque estamos en tiempos donde la sinergia en múltiples áreas como pueden ser grupos de investigación o de trabajo internacionales va marcando tendencia y será necesario que los sistemas puedan brindar soluciones en esa misma dirección.