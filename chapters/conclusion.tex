Resulta cada vez más frecuente la creación de disciplinas y ramificaciones de las ciencias que vienen para resolver problemáticas puntuales. La Recuperación de Información puede ser un buen ejemplo de ello y, como muchos otros campos, se enfocan en atacar el problema generando modelos basados en conocimientos previos para concluir en una posible solución. Actualmente, podemos ver cómo la interrelación con otras áreas de estudio e investigación, como puede ser la Minería de Datos o la Inteligencia Artificial, aportan soluciones integrales para mejorar resultados, y valor a nuevos desafíos que van surgiendo en el campo. \\

Tomando como caso de estudio a los motores de búsquedas, vemos que éstos deben lidiar con problemas adicionales, a comparación con un simple sistema de recuperación; es decir, deben dar respuesta a búsquedas, complejas, sobre colecciones de objetos con datos estructurados, semi-estructurados y no estructurados que presentan un dinamismo de cambio sin precedentes. El hecho que se pueda recuperar datos con diversos formatos genera mayores desafíos para todo el proceso de indexación y recuperación. A lo largo de éste trabajo hemos estudiado distintas herramientas para llevar a cabo lo anterior. Un claro ejemplo de porqué la Recuperación de Información es un campo complejo es, sin dudas, la existencia de la World Wide Web. Para éste caso, hemos estudiado ciertas herramientas y algoritmos para poder \textit{afrontar} esta enorme y, particularmente, diversa colección de datos. \\

Por último, un acercamiento en el estado del arte con los sistemas recomendadores, los  cuales aportan una mejor experiencia para el usuario satisfaciendo necesidades de información de forma implícita y aprovechando de la relación entre usuarios para entregar una respuesta con mayor precisión. \\

Dadas las magnitudes de distintos repositorios de información, un enfoque que no fue tratado en éste trabajo pero sí creemos que debe investigarse en profundidad es la \textit{recuperación colaborativa}, más comúnmente abreviada \textit{CIR}. CIR es el estudio de sistemas y prácticas que permiten la colaboración entre personas durante el proceso de búsqueda, y se caracterizan por la colaboración implícita y asíncrona con el fin de mejorar el proceso de recuperación tradicional. El motivo de éstos sistemas es mejorar el rendimiento de un grupo de usuarios que buscan en conjunto para satisfacer una necesidad compartida de información; ésto, por ejemplo, impide realizar búsquedas repetidas o no leer documentos que se encuentren marcados por otro participante del grupo. Entre los científicos que estudian éste enfoque, se encuentran Colum Foley Alan F. Smeaton, entre otros.