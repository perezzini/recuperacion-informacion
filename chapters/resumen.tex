En el presente trabajo realizaremos una revisión en el campo de la recuperación de información, tomando la definición y el concepto de los compatriotas que han estado estudiando e investigando en el área por varios años: Ricardo Baeza-Yates y Berthier Ribeiro-Neto. Para luego poder analizar algunas de la etapas del proceso que realizan los sistemas de recuperación de información: el procesamiento y la indexación de los documentos, los modelos más influyentes para la implementación de estos sistemas abordando la representación de los objetos de información y la forma de recuperar esos objetos ante una consulta. También algunas medidas que nos permitan determinar la calidad de la recuperación que tiene un sistema. \par
Además un poco de historia sobre la Web y algunos conceptos y algoritmos desarrollados para analizar, recolectar, indexar, recuperar y visualizar información de esa gran red de redes. Donde se deben lidiar con problemas extras como la heterogeneidad de los datos y la generación de información de fuentes poco confiables.\par
Por último un acercamiento a los sistemas recomendadores que tratan de recuperar información relevante para cada usuario y ponerla a su disposición mediante el uso de técnicas en inteligencia artificial. Estas aplicaciones resultan cada vez más convenientes en nuestras vidas cotidianas.
