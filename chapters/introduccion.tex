\section{Recuperación de Información}
	La recuperación de información ocupa un gran área de interés en las Ciencias de la Computación. Brinda herramientas las cuales utilizamos de forma cotidiana y que, en ciertas situaciones, dependemos en su totalidad. Estas herramientas nos acercan de forma eficaz a información de nuestro interés. Cabe destacar que, sin el desarrollo y la existencia de ésta rama, los usuarios comunes y corrientes no serían capaces de obtener la información que necesitan dado el tamaño de las fuentes de información que va en aumento día a día. Existen diversas definiciones de recuperación de información; una breve pero contundente descripción es la siguiente: \textit{La recuperación de información trata con la representación, almacenamiento, organización y acceso a ítems de información tales como documentos, páginas Web, catálogos online, datos estructurados y no estructurados, objetos multimedia. La representación y organización de los ítems de información tienen que ser tales como para proveer a los usuarios un fácil acceso a información de su interés} \cite{baeza1999}.
	
	\subsection{Recuperación de Información vs. Recuperación de Datos}
	Dada una consulta del usuario que representa el input necesario de un sistema de recuperación de información, el objetivo principal de éste es darle al usuario acceso rápido a un conjunto de documentos con información útil ordenados de acuerdo a su relevancia.  \\
	Por otro lado un sistema de recuperación de datos sólo se encarga de determinar el conjunto de documentos que cumple con cierta proposición sobre los datos del documento.
	En el primer caso, el problema es más complejo porque además de saber como recuperar los objetos de información se tiene que determinar cuales objetos resultarán más importantes que otros para el usuario de acuerdo a los términos utilizados en la consulta, los cuales tienen cierta ambigüedad derivada del lenguaje natural, debido a esto la respuesta tendrá un grado de imprecisión y es posible que se recuperen parcialmente los documentos que el usuario considera importantes. \\
	En cambio la respuesta de un sistema de recuperación de datos es exacta de acuerdo a la consulta que se le realiza en la mayoría de los casos utilizando lenguajes altamente estructurados que lo vuelve mucho más sensible a los errores que pueda tener el input del usuario.
	
	\subsection{El proceso de recuperación y \textit{ranking}}
		Supongamos que tenemos una colección de documentos de la cual, un usuario, quiere obtener información relevante. Los sistemas de 	recuperación de información deben, primeramente, procesar estos documentos: eliminar palabras \textit{stop--words}, \textit{stemming}, \textit{normalización}, entre otras. Este proceso se conoce, generalmente, como \textbf{procesamiento de documentos} y se analiza detalladamente en el Capítulo XX. Resumiendo, este proceso tiene como finalidad extraer \textit{términos} para, luego, crear un índice. Estos términos se consideran palabras claves (\textit{keywords}) dentro de la colección. \par

		Una vez que se extraen los términos de los documentos, se procede a crear un índice del texto. Para ésto, se utilizan diferentes estructuras de datos; pero la más popular es llamada \textbf{índice invertido}. Esta tarea se denomina \textit{indexado} y debe ser realizada \textit{offline}, antes de que el sistema de recuperación de información pueda procesar cualquier consulta. La implementación de éste proceso es de extrema importancia; ya que, dadas largas colecciones de documentos, computarlo puede tomar tiempo considerable. \par

		Ahora, teniendo la colección de documentos completamente indexada, el proceso de recuperación de información puede comenzar. Este consiste, básicamente, en que el usuario ingresa al sistema una consulta; la cual está, en sistemas de recuperación de información modernos, escrita en lenguaje natural o en uno cercano al mismo. A continuación, el sistema aplica una serie de procedimientos de análisis de texto a la consulta. Estos procedimientos son similares a los que se realizan en el procesamiento de texto anterior: eliminación de \textit{stop--words}, correcciones de errores de ortografías, entre otros. Es decir, al igual que a los documentos de la colección, se le extraen los términos. El sistema, también, puede presentar al usuario consultas sugeridas, para que el mismo confirme. Una vez procesada la consulta, el sistema recupera un conjunto de documentos. Estos documentos contienen los términos que han sido extraídos de la consulta ingresada por el usuario. Obtener éstos documentos es una tarea rápida debido al índice construido de antemano. El proceso anterior es llamado \textit{proceso de recuperación}. \par

		Luego de realizar el proceso de recuperación, se deben ordenar los documentos dentro del conjunto obtenido. Es una tarea crítica ya que se encuentra en juego la conformidad del usuario y, así, el retorno del mismo al sistema. Este ordenamiento de documentos es realizado mediante funciones de similitud, las cuales se detallan en el Capitulo XX.